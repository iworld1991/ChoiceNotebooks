\documentclass{handout}
\usepackage{handoutSetup}  
\usepackage{import}

\setboolean{SpaceBetweenAnswers}{true}  % If for an exam where students will be writing on the exam
\setboolean{SpaceBetweenAnswers}{false} % If for a problem set or comprehensive exam where they will not be writing on the paper   

%\setboolean{AnswersTF}{true}

\begin{document}
\thispagestyle{empty}

\centerline{\Large Problem Set 6}\medskip
\centerline{\large Intertemporal Choice}\medskip\medskip

\label{qSub:aDiff} This problem set asks you to start with \href{https://lectures.quantecon.org/py/optgrowth.html}{Exercise 1} of the QuantEcon lecture on Stochastic Optimal Growth: \url{https://lectures.quantecon.org/py/optgrowth.html}. The original exercise is to simulate the models' dynamics for three different discount factors (hence three different policy rules). \\

Modify the solution code to simulate 500 periods for each of the following cases (keeping all other parameters at their default values):
		\begin{enumerate}
			\item Capital's share of income
			\begin{enumerate}
				\item $\alpha$ = 0.1
				\item $\alpha$ = 0.4
				\item $\alpha$ = 0.7
			\end{enumerate}
			\item The coefficient of relative risk aversion.  (Hint: For the non-logarithmic utility case, you will need to start by redefining the utility function to be of the CRRA form rather than logarithmic)
			\begin{enumerate}
				\item $\rho$ = 1.0 (the baseline case)
				\item $\rho$ = 2.0
				\item $\rho$ = 5.0
				\item $\rho$ = 20.0
			\end{enumerate}
		\end{enumerate}

                For each of these cases, your goal is to estimate the average simulated capital stock levels for different values of the parameter ($\alpha$, or $\rho$). For each choice of the parameter value, you should estimate the `steady state` as the average value for simulations of the model from period 200 onward).


\end{document}
